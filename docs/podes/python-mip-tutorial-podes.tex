\documentclass[a4paper,11pt,fleqn]{article}
\usepackage{podes-template}

%pacotes adicionais
\usepackage[linesnumbered, algoruled, vlined, portuguese]{algorithm2e}
\usepackage{listings}
\lstset
{ %Formatting for code in appendix
	language=Python,
	numbers=left,
	stepnumber=1,
	showstringspaces=false,
	tabsize=1,
	breaklines=true,
	breakatwhitespace=false,
}

%%%%%%%%%%%%%%%%%%%%%%%%%%%%%%%%%%% %%%%%%%%%%%%%%%%%%%%%%%%%%%%%%%%%%%%%%%


%título do artigo
\title{Desenvolvendo Resolvedores de Programação Linear Inteira Mista em Python usando o pacote Python-MIP$^1$} 

%define os autores
\author{
 \name{Haroldo G. Santos\authortag{a}\corresponding{haroldo@ufop.edu.br}}, 
 \name{Túlio A.M. Toffolo\authortag{a}} \\
 \authortag{a}
 \institute{Instituto de Ciências Exatas e Biológicas, Departamento de Computação\\ Universidade Federal de Ouro Preto, Ouro Preto, MG, Brasil}
}

\authorrunning{Santos \& Toffolo}


\begin{document}


\maketitle


\begin{resumo}
O pacote Python MIP oferece ferramentas para a modelagem e resolução de Problemas de Programação Inteira Mista em Python. Além de uma linguagem de modelagem de alto nível, o pacote permite o desenvolvimento de resolvedores avançados, habilitando comunicação bidirecional com o resolvedor durante o processo de busca. Neste tutorial desenvolveremos resolvedores de Programação Linear Inteira Mista para o Problema do Caixeiro Viajante. Iniciando com um resolvedor simples baseado em uma formulação compacta iremos evoluir para um resolvedor que combina heurísticas, planos de corte e estratégias de ramificação para a resolução eficaz de problemas de grande porte.

\end{resumo}

\begin{palavras}
Primeira, Segunda, Terceira, Quarta.
\end{palavras}

\begin{abstract}
This document presents the format for full papers to be published in the journal Pesquisa Operacional para o Desenvolvimento. The Abstract must not exceed 150 words.
\end{abstract}

\begin{keywords}
First keyword, Second keyword, Third keyword, Last keyword. 
\end{keywords}


\newpage
\thispagestyle{defaultPage}

Python-MIP é um pacote para modelagem e resolução de Problemas de
Programação Linear Inteira Mista (PLIM) \citep{Wolsey1998} em Python.
O projeto do pacote foi feito com o objetivo de desenvolver uma ferramenta
que atendesse os seguintes requisitos:

\begin{enumerate}
	\item clareza de código e modelagem de alto nível
	\item alto desempenho
	\item extensibilidade e configurabilidade
\end{enumerate}

Tradicionalmente, os objetivos 1 e 2 foram considerados conflitantes.
Até recentemente, as opções mais recomendada para os interessados
em 1 eram linguagens algébricas de alto nível como AMPL \citep{Fourer1987}.
A obtenção de desempenho máximo costumava requerer o uso de linguagens
de mais baixo nível como C\citep{Johnson1991a}. Resolvedores estado-da-arte
como o CPLEX foram escritos nessa linguagem\citep{Bixby2002}. Desse
modo, a biblioteca completa de funções estava originalmente 
disponível somente nela. Recentemente, soluções como JuMP\citep{Dunning2015}
demonstraram que os objetivos 1 e 2 não são necessariamente conflitantes:
linguagens de alto nível como Julia juntamente com compiladores \emph{just-in-time}
permitem o desenvolvimento rápido de resolvedores que apresentam alto
desempenho. O objetivo do projeto Python-MIP é o desenvolvimento de
um pacote de Programação Linear Inteira Mista para a linguagem Python
que atenda plenamente os requisitos 1-3.

Pesquisas recentes mostram que Python está se tornando a linguagem
mais popular da atualidade\citep{pythonEconomist2018}. O projeto Python-MIP
foi primariamente inspirado em dois projetos de código aberto para
programação linear inteira em Python. O primeiro é o PuLP \citep{Mitchell2009},
que oferece uma linguagem de modelagem de alto nível e interface para
vários resolvedores comerciais e de código aberto. Recursos que requerem 
uma integração maior com o resolvedor, como geração dinâmica de planos de cortes, não estão
disponíveis neste pacote. O pacote CyLP, por outro lado, suporta geração
dinâmica de planos de corte mas não oferece uma linguagem de modelagem
de alto nível\citep{Towhidi2016} e somente suporta o resolvedor COIN-OR
CBC\citep{Forrest2005}. O pacote Python-MIP foi criado com o objetivo
de prover a funcionalidade dos dois pacotes com máximo desempenho.
A escrita de um um novo pacote de programação linear inteira em Python
também permite que recursos relativamente novos da linguagem, como
a tipagem estática e a comunicação direta com bibliotecas nativas
(Python CFFI) sejam utilizados extensivamente no código.

Neste tutorial desenvolveremos versões sucessivamente mais sofisticadas
de um resolvedor para o clássico problema do Caixeiro Viajante\citep{Applegate2006} no
pacote Python-MIP. Enquanto na primeira versão a comunicação com o resolver 
somente ocorre no momento em que o modelo criado é informado e na coleta dos resultados, 
a versão final utiliza comunicação bidirecional com o resolvedor durante o processo de busca para o tratamento
de uma formulação com um número exponencial de restrições em um método de \emph{branch-\&-cut}. Resultados experimentais são apresentados na seção XXX 
para demonstrar os ganhos substanciais de desempenho que podem ser obtidos com essa última versão.

\section{Aplicação: Problema do Caixeiro Viajante}

O problema do caixeiro viajante consiste em: dada uma malha viária
e um conjunto de pontos que devem ser visitados, encontrar uma rota
de custo mínimo (tempo ou distância, usualmente) que inclua todos os pontos percorrendo-os exatamente uma vez. Formalmente, temos como dados de entrada um grafo direcionado $G=(V,A)$ com custos associadas aos arcos:
\begin{description}
	\item [{$V$}] conjunto de vértices numerados sequencialmente a partir
	de 0
	\item [{$c_{(i,j)}$}] custo de percorrer a ligação $(i, j) \in V \times V $
\end{description}

Em todas as formulações que serão apresentadas, utilizaremos as seguintes
variáveis binárias de decisão que representam a escolha dos arcos
que compõe a rota:

\[
x_{(i,j)}=\begin{cases}
1 & \textrm{se o arco }(i,j)\textrm{ foi escolhido para a rota}\\
0 & \textrm{caso contrário}
\end{cases}
\]

Como exemplo, considere o mapa da Figura \ref{figG} que inclui algumas cidades turísticas da Bélgica:

\begin{figure}
	\begin{centering}
		\includegraphics[width=0.6\textwidth]{belgium-tourism-14.png}
		\par\end{centering}
	\caption{14 cidades turísticas da Bélgica}	
	\label{figG}
\end{figure}


\subsection{Uma Formulação Compacta}

O problema do caixeiro viajante pode ser modelado utilizando-se uma
formulação compacta, isto é, uma formulação com um número polinomial
de variáveis e restrições. Formulações desse tipo, apesar de usualmente não serem a melhor opção de resolução em termos de desempenho para problemas
deste tipo, são convenientes para uma primeira abordagem pois podem
ser facilmente inseridas de uma vez só como entrada para um software
resolvedor. A formulação abaixo foi proposta por \cite{Miller1960} nos
anos 60:

\begin{align}
    \textrm{Minimize: }   & \nonumber \\
    &  \sum_{i \in I, j \in I} c_{i,j} \ldotp x_{i,j} \\
    \textrm{Subject to: }   & \nonumber \\
    & \sum_{j \in V \setminus \{i\}} x_{i,j} = 1 \,\,\, \forall i \in V \label{eq:in}  \\
    & \sum_{i \in V \setminus \{j\}} x_{i,j} = 1 \,\,\, \forall j \in V \label{eq:out} \\
    & y_{i} -(n+1)\ldotp x_{i,j} \geq y_{j} -n  \,\,\, \forall i \in V\setminus \{0\}, j \in V\setminus \{0,i\} \label{eq:st1} \\
    & x_{i,j} \in \{0,1\} \,\,\, \forall i \in V, j \in V \\
    & y_i \geq 0 \,\,\, \forall i \in V 
\end{align}

As equações (\ref{eq:in}) e (\ref{eq:out}) garantem que cada vértice
é visitado somente uma vez enquanto variáveis auxiliares $y_{i}$
são utilizadas nas restrições (\ref{eq:st1}) para garantir que uma vez
que um arco $(i,j)$ seja selecionado, o valor de $y_{j}$ seja maior
do que o valor de $y_{i}$ em uma unidade. Essa propriedade é garantida
para todos os vértices exceto o vértice 1 que é arbitrariamente selecionado
como origem de modo a evitar a construção de sub-rotas desconectadas
como no exemplo da Figura \ref{figSub}, onde os valores das variáveis
$x_{(i,j)}$ indicados nos arcos representam uma solução viável caso
somente as restrições (\ref{eq:in}) e (\ref{eq:out}) fossem consideradas.

\begin{figure}
	\begin{centering}
		\includegraphics[width=0.6\textwidth]{belgium-tourism-14-subtour.png}
		\par\end{centering}
	\caption{Rotas desconectadas da origem}
	\label{figSub}
	
\end{figure}

A seguir temos um exemplo completo de um resolvedor em Python-MIP para o problema do caixeiro viajante para o mapa Figura \ref{figG}, onde o resolvedor utilizará a formulação compacta descrita anteriormente:

{\small
\lstinputlisting[breaklines]{tsp-compact.py}
}

Nas linhas 4 nomeamos nossos pontos turísticos. Nas linhas 10-23 informamos a distância entre cada par de cidades $(i, j)$ onde $i<j$, visto que em nosso exemplo consideramos que a distância de ida e volta é igual. Convertemos a matriz triangular \texttt{dists} em uma matriz completa nas linhas 27-20.

A linha 30 cria o modelo de programação linear inteira. As variáveis do modelo são criadas nas linhas 32 e 34 utilizando o método \texttt{add\_var} em nosso modelo \texttt{m}. Durante a criação das restrições, será necessário referenciar as variáveis criadas. Por isso, utilizamos os dicionários \texttt{x} para mapear cada arco do grafo com sua respectiva variável binária e \texttt{y} para mapear cada nó com sua respectiva variável auxiliar contínua para eliminação de sub-rotas.

A função objetivo que minimiza o custo total dos arcos selecionados é informada na linha 36. Nesse caso, para cada arco multiplicamos sua respectiva variável binária de seleção pela distância do arco armazenada em \texttt{c}.

As restrições são criadas nas linhas 38-45. Em todos os casos utilizamos o operador \texttt{+=} sobre o modelo \texttt{m} para adicionar restrições lineares. Note que assim como na função objetivo, o somatório é efetuado com a função \texttt{xsum}. Esta função é similar a função \texttt{sum} disponível na linguagem Python mas otimizada para a situação específica de escrita de restrições lineares no pacote Python-MIP\@. 

A linha 47 dispara a otimização do modelo. Na linha 49 verificamos se uma solução viável foi encontrada e se positivo a escrevemos nas linhas 50-58.

Os resolvedores de programação linear inteira executam uma busca em árvore onde são utilizados limites para a poda de nós. O limite superior é obtido a partir de qualquer solução viável que for encontrada durante a busca e o limite inferior corresponde ao custo obtido com a resolução do problema com as restrições de integralidade das variáveis binárias relaxadas. No nosso caso considerando domínio das variáveis $x$ como contínuo entre 0 e 1. A formulação aqui utilizada tem uma grave deficiência: o limite inferior por ela produzido é distante do custo ótimo da solução. Desse modo, o desempenho dos resolvedores de programação linear inteira em sua resolução é bastante pobre. 

Um desempenho muito melhor pode ser obtido com a inserção das seguintes desigualdades:

\begin{equation}
\sum_{(i,j) \in V \times V : i\in S \land j \notin S} x_{(i,j)} \leq |S|-1 \,\,\, \forall S \subset V 
\end{equation}

O problema com as desigualdades acima é que elas devem ser geradas para cada subconjunto $S$ de vértices do grafo, ou seja, para uma grafo com $n$ vértices temos $2^n-1$ subconjuntos não vazios. Inseri-las no modelo inicial é inviável exceto para instâncias pequenas. Uma solução para esse problema é o método dos planos de corte \citep{Dantzig54}, onde somente as restrições \emph{necessárias} são inseridas. O pacote Python-MIP permite uma comunicação bi-direcional com o resolvedor para que as restrições necessárias sejam inseridas \emph{durante} a busca. Para isso precisamos criar uma classe derivada da classe \texttt{ConstrsGenerator} que implemente o método \texttt{generate\_constrs}. Esse método recebe como parâmetro um modelo, onde a solução a solução corrente pode ser consultada e restrições adicionais podem ser inseridas na medida do necessário. 

\subsection{Geração de Planos de Corte}

O código abaixo implementa um gerador dinâmico de restrições de eliminação de sub-rotas para nossa formulação compacta previamente implementada. 

{\small
\begin{lstlisting}
import networkx as nx

class SubTourCutGenerator(ConstrsGenerator):
    def __init__(self, Fl: List[Tuple[int, int]]):
        self.F = Fl

    def generate_constrs(self, model: Model):
        G = nx.DiGraph()
        r = [(v, v.x) for v in model.vars if v.name.startswith('x(')]
        U = [int(v.name.split('(')[1].split(',')[0]) for v, f in r]
        V = [int(v.name.split(')')[0].split(',')[1]) for v, f in r]
        cp = CutPool()
        for i in range(len(U)):
            G.add_edge(U[i], V[i], capacity=r[i][1])
        for (u, v) in F:
            if u not in U or v not in V:
                continue
            val, (S, NS) = nx.minimum_cut(G, u, v)
            if val <= 0.99:
                arcsInS = [(v, f) for i, (v, f) in enumerate(r)
                           if U[i] in S and V[i] in S]
                if sum(f for v, f in arcsInS) >= (len(S)-1)+1e-4:
                    cut = xsum(1.0*v for v, fm in arcsInS) <= len(S)-1
                    cp.add(cut)
                    if len(cp.cuts) > 256:
                        for cut in cp.cuts:
                            model += cut
                        return
        for cut in cp.cuts:
            model += cut
        return
\end{lstlisting}}

Na criação de nosso gerador de cortes informamos uma lista \texttt{Fl} de pares $(i,j)$ de vértices cuja conectividade deve ser checada em toda solução gerada. No método \texttt{generate\_constrs} consultamos as variáveis do modelo (\texttt{model.vars}) e identificamos a qual arco cada variável se refere considerando o nome das variáveis (linhas 22 e  23), utilizando o seu valor na solução (propriedade \texttt{x}) para construção do grafo onde iremos procurar sub-rotas desconexas para geração das restrições (6). A identificação dessas sub-rotas é feita resolvendo-se o problema do corte mínimo, com o algoritmo que está disponível no pacote \texttt{networkx} (linha 18). A descoberta de sub-rotas desconexas (linhas 19) gera a restrição de restrições (cortes) que são primeiramente armazenados em um conjunto (linha 24) e posteriormente adicionados ao modelo. Separamos esses dois passos pois restrições repetidas podem ser geradas. Na linha 25 inserimos um critério de parada para a geração dos cortes: caso um número suficientemente grande já tenha sido gerado, inserimos os mesmos no modelo e prosseguimos a otimização sem procurar por cortes adicionais. Nesse ponto convém ressaltar a função dos cortes e sua relação com o restante do modelo. Como a formulação compacta que estamos utilizando define completamente o problema, a adição de cortes é opcional, ou seja, somente feita para melhorar o desempenho do modelo. A inserção de um número muito grande de cortes por iteração pode gerar o efeito indesejado de perda de desempenho na resolução. Para utilizamos nosso gerador de cortes com a formulação anterior basta atribuir à propriedade \texttt{cuts\_generator} um objeto da nossa classe \texttt{SubTourCutGenerator} antes da otimização do modelo.

\subsection{Integração com heurísticas}

Resolvedores de programação linear inteira iniciam o processo de resolução computando uma solução possivelmente fracionária, obtida através da relaxação do problema. Em instâncias difíceis, a obtenção da primeira solução \emph{inteira} válida pode requerer a exploração de um grande número de nós na árvore de busca. Para essas instâncias, muitas vezes uma heurística simples e rápida pode ser utilizada para geração de uma solução inicial. No pacote Python-MIP soluções iniciais podem ser facilmente informadas ao resolvedor através da propriedade \texttt{start} do modelo que recebe uma lista de pares \texttt{(x, v)} onde \texttt{x} é uma referência para uma variável de decisão e \texttt{v} o seu valor na solução factível. 

O código abaixo demonstra a utilização de um algoritmo construtivo e de uma metaheurística de busca local para construção de uma solução inicial factível para nosso modelo.
 
{\small
\begin{lstlisting}
seq = [0, max((c[0][j], j) for j in V)[1]] + [0]
Vout = V-set(seq)
while Vout:
    (j, p) = min([(c[seq[p]][j] + c[j][seq[p+1]], (j, p)) for j, p in
                  product(Vout, range(len(seq)-1))])[1]

    seq = seq[:p+1]+[j]+seq[p+1:]    
    Vout = Vout - {j}


def delta(d: List[List[float]], S: List[int], p1: int, p2: int) -> float:
    p1, p2 = min(p1, p2), max(p1, p2)
    e1, e2 = S[p1], S[p2]
    if p1 == p2:
        return 0
    elif abs(p1-p2) == 1:
        return ((d[S[p1-1]][e2] + d[e2][e1] + d[e1][S[p2+1]])
                - (d[S[p1-1]][e1] + d[e1][e2] + d[e2][S[p2+1]]))
    else:
        return (
        (d[S[p1-1]][e2] + d[e2][S[p1+1]] + d[S[p2-1]][e1] + d[e1][S[p2+1]])
        - (d[S[p1-1]][e1] + d[e1][S[p1+1]] + d[S[p2-1]][e2] + d[e2][S[p2+1]]))

L = [cost for i in range(50)]
sl, cur_cost, best = seq.copy(), cost, cost
for it in range(int(1e7)):
    (i, j) = rnd.randint(1, len(sl)-2), rnd.randint(1, len(sl)-2)
    dlt = delta(c, sl, i, j)
    if cur_cost + dlt <= L[it % len(L)]:
        sl[i], sl[j], cur_cost = sl[j], sl[i], cur_cost + dlt
        if cur_cost < best:
            seq, best = sl.copy(), cur_cost
    L[it % len(L)] = cur_cost
    
m.start = [(x[seq[i]][seq[i+1]], 1) for i in range(len(seq)-1)]    
\end{lstlisting}}

Nosso algoritmo construtivo (linhas 262-269) será o algoritmo da inserção mais barata: dada uma rota que não inclua todos as cidades, na inserção de uma nova cidade verificamos o custo de inserir cada cidade ainda fora da rota (elemento \texttt{j} conjunto \texttt{Vout} em cada posição intermediária possível (\texttt{p}) e selecionamos a opção mais barata.

Para melhoria da nossa solução inicial utilizaremos a metaheurística baseada em busca local Late Acceptance Hill Climbing \citep{burke2017} por sua simplicidade. O gargalo de métodos de busca local usualmente é a avaliação do custo da solução resultante da aplicação de dado movimento. Por isso, nas linhas 11-22 incluímos uma função que dada uma solução de entrada (sequência de cidades) \texttt{s} e duas posições dessa sequência, \texttt{p1} e \texttt{p2} calcula em \emph{tempo constante} a variação de custo que será obtida. Dessa forma, podemos executar rapidamente um grande número de movimentos cuja aceitação é controlada pelo arcabouço da metaheurística que é implementada nas linhas 25-33. Finalmente, informamos as variáveis de decisão relacionadas aos arcos existentes na melhor solução encontrada na linha 35.


		
\bibliography{pmip-podes}
\bibliographystyle{podes-bibstyle}


\end{document}
